\section{Decision-Making Theory} \label{sec:theory}
Human decision makers in time-constrained situations such as fire fighting tend to evaluate options serially. They attempt to find one viable plan that should work rather than attempting to generate and compare numerous plans in parallel. This theory has been described by Klein and Calderwood~\cite{KleinCalderwood} as Recognition Primed Decision-making (RPD). Initially, human experts look for similarities to previous situations that bring up goals that were relevant, things that were important to monitor, and actions that were possible to pursue. Then, they go through a process of mental simulation to consider whether the actions would work also for the case at hand. They also assess the ongoing situation, looking for violations and confirmations of expectancies, which may require reframing the situation. Klein and Calderwood suggested also that ``displays and interfaces can be centered on decisions rather than around data flows'', emphasizing that systems can be built to enhance the decision process. 

Similarly, the contextual control model (COCOM) describes how people rely on context when making decisions~\cite{hollnagel2005joint}. This theory recognizes that humans sometimes act with plans of lower quality, relying on the environment to make decisions of next steps opportunistically without having a whole plan to reach the end goal. The quality of their control of the situation can be described as scrambled, opportunistic, tactical, or strategic. The scrambled mode refers to decisions made without any idea of what to do. In the opportunistic mode, people rely on cues in the local context to decide their next action. In tactical mode, they have or get an idea of how to achieve their goal --- a plan. In strategic mode, the plan includes coordination with other simultaneous goals. The goal of our system is to lift the quality of control from being opportunistic (as in the current workflow) to being strategic, thus enabling improved decision-making capabilities.

Turning to the granularity of plans, the Extended Control Model (ECOM)~\cite{hollnagel2005joint} describes plans in terms of a tactical level (setting goals), monitoring (making plans and overseeing plans), regulating (managing local resources), and tracking (performing and adjusting actions). This theory can be used to apprise what kind of planning support a system provides.  Moreover, it has been argued by Lundberg \etal\ that in areas such as emergency response it is important to support resiliency: ``Rather than merely selecting a response from a ready-made table, [the system] must adapt and create a suitable response; either by following ready-made plans for adaptation or by making sense of the situation and create responses during the unfolding event''~\cite{Lundberg2012}. Thus, in addition to supporting specific responses that can be foreseen, the system should also support work outside of specific prepared means.
