\section{Related Work} \label{sec:relatedwork}
\noindent {\bfseries Emergency management.} Much of the visualization-oriented work that is published in the general field of emergency management is concerned with evacuation planning well before any rescue operation has to be performed. Notable work was performed by Reddy \etal, which enables analyzing possible bottlenecks of escape routes~\cite{EuroVA12:13-17:2012}. While algorithms from these areas could be utilized to our benefit, they usually assume perfect walking conditions and a constant structural layout of the building. Ribarsky \etal\ presented a system organizing first responders in intact structures~\cite{Ribarsky:2010}. Kim \etal\ developed a system enhancing the situational awareness of responders using a mobile visual analytics tool~\cite{Kim:2008}. Another related area is the research on visual analysis-supported ensemble steering. While these techniques use ensembles and the visualization to reduce the impact of uncertainty in the input parameters, our system helps the operator to find the best trade-off between the various runs. Ribi\v{c}i\'c \etal\ investigated steering ensembles of flooding simulations using visual analysis~\cite{6280550}. They showed that visual analysis is a valid and useful method for experts to interpret this kind of data. Their idea of generating different measures about a particular flooding inspired our approach to do computation of path ensembles. 

Many existing planning systems deployed nowadays in USAR scenarios are based on 2D representations~\cite{kleiner_et_al_ssrr09,KohlbrecherMeyerStrykKlingaufFlexibleSlamSystem2011,Pellenz2009SMU}. Given the 2D map of the environment, one common approach to path planning is to plan the shortest trajectory and to follow this trajectory stepwise. In harsh environments, however, not the shortest but the safest path can be desirable. Wirth~\etal\ introduced an exploration strategy and path planner that utilizes occupancy grid maps when planning to several targets at the same time~\cite{Wirth2007ETA1}. Consequently, the method selects the safest alternative consisting of target location and path to reach the target. Preliminary extensions towards exploration in 3D were introduced by Dornhege and Kleiner~\cite{dornhege2011frontier}.

%\noindent\textbf{Visual analytics.} There has been work investigating the sense-making process when using visual analytics tools. Gotz \etal\ investigated how to measure insight provenance, a critical measure to judge the quality of reasoning~\cite{Gotz2009}. Roberts presented a state-of-the-art report on how to apply the visual analytics models and techniques using coordinated and multiple views~\cite{roberts2007multipleviews}. Keim \etal\ provide a good overview about the current developments in the field~\cite{keim2010mastering}.

\noindent {\bfseries Point cloud visualization.} As the real-time availability of high resolution point clouds has increased in recent years, there has been much research on rendering techniques for this kind of data. Basic rendering capabilities are provided by the widely used Point Cloud Library which provides a wide array of functionality~\cite{Rusu11ICRA}. Furthermore, there has been work by Richter \etal\ who used a level-of-detail structure to render massive point clouds at high frame rates~\cite{Richter:2010:ORV:1811158.1811178}. Xu \etal\ showed that non-photorealistic rendering techniques can be applied to point cloud data and the resulting contour lines in their rendering inspired our rendering algorithm~\cite{conf/npar/XuC04}. More recently Pintus \etal\ presented a rendering algorithm that enhances features of unstructured point clouds in real time without preprocessing~\cite{Pintus:2011:RRM:2384495.2384513}.
